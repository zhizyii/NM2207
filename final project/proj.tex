% Options for packages loaded elsewhere
\PassOptionsToPackage{unicode}{hyperref}
\PassOptionsToPackage{hyphens}{url}
%
\documentclass[
]{article}
\usepackage{amsmath,amssymb}
\usepackage{iftex}
\ifPDFTeX
  \usepackage[T1]{fontenc}
  \usepackage[utf8]{inputenc}
  \usepackage{textcomp} % provide euro and other symbols
\else % if luatex or xetex
  \usepackage{unicode-math} % this also loads fontspec
  \defaultfontfeatures{Scale=MatchLowercase}
  \defaultfontfeatures[\rmfamily]{Ligatures=TeX,Scale=1}
\fi
\usepackage{lmodern}
\ifPDFTeX\else
  % xetex/luatex font selection
\fi
% Use upquote if available, for straight quotes in verbatim environments
\IfFileExists{upquote.sty}{\usepackage{upquote}}{}
\IfFileExists{microtype.sty}{% use microtype if available
  \usepackage[]{microtype}
  \UseMicrotypeSet[protrusion]{basicmath} % disable protrusion for tt fonts
}{}
\makeatletter
\@ifundefined{KOMAClassName}{% if non-KOMA class
  \IfFileExists{parskip.sty}{%
    \usepackage{parskip}
  }{% else
    \setlength{\parindent}{0pt}
    \setlength{\parskip}{6pt plus 2pt minus 1pt}}
}{% if KOMA class
  \KOMAoptions{parskip=half}}
\makeatother
\usepackage{xcolor}
\usepackage[margin=1in]{geometry}
\usepackage{color}
\usepackage{fancyvrb}
\newcommand{\VerbBar}{|}
\newcommand{\VERB}{\Verb[commandchars=\\\{\}]}
\DefineVerbatimEnvironment{Highlighting}{Verbatim}{commandchars=\\\{\}}
% Add ',fontsize=\small' for more characters per line
\usepackage{framed}
\definecolor{shadecolor}{RGB}{248,248,248}
\newenvironment{Shaded}{\begin{snugshade}}{\end{snugshade}}
\newcommand{\AlertTok}[1]{\textcolor[rgb]{0.94,0.16,0.16}{#1}}
\newcommand{\AnnotationTok}[1]{\textcolor[rgb]{0.56,0.35,0.01}{\textbf{\textit{#1}}}}
\newcommand{\AttributeTok}[1]{\textcolor[rgb]{0.13,0.29,0.53}{#1}}
\newcommand{\BaseNTok}[1]{\textcolor[rgb]{0.00,0.00,0.81}{#1}}
\newcommand{\BuiltInTok}[1]{#1}
\newcommand{\CharTok}[1]{\textcolor[rgb]{0.31,0.60,0.02}{#1}}
\newcommand{\CommentTok}[1]{\textcolor[rgb]{0.56,0.35,0.01}{\textit{#1}}}
\newcommand{\CommentVarTok}[1]{\textcolor[rgb]{0.56,0.35,0.01}{\textbf{\textit{#1}}}}
\newcommand{\ConstantTok}[1]{\textcolor[rgb]{0.56,0.35,0.01}{#1}}
\newcommand{\ControlFlowTok}[1]{\textcolor[rgb]{0.13,0.29,0.53}{\textbf{#1}}}
\newcommand{\DataTypeTok}[1]{\textcolor[rgb]{0.13,0.29,0.53}{#1}}
\newcommand{\DecValTok}[1]{\textcolor[rgb]{0.00,0.00,0.81}{#1}}
\newcommand{\DocumentationTok}[1]{\textcolor[rgb]{0.56,0.35,0.01}{\textbf{\textit{#1}}}}
\newcommand{\ErrorTok}[1]{\textcolor[rgb]{0.64,0.00,0.00}{\textbf{#1}}}
\newcommand{\ExtensionTok}[1]{#1}
\newcommand{\FloatTok}[1]{\textcolor[rgb]{0.00,0.00,0.81}{#1}}
\newcommand{\FunctionTok}[1]{\textcolor[rgb]{0.13,0.29,0.53}{\textbf{#1}}}
\newcommand{\ImportTok}[1]{#1}
\newcommand{\InformationTok}[1]{\textcolor[rgb]{0.56,0.35,0.01}{\textbf{\textit{#1}}}}
\newcommand{\KeywordTok}[1]{\textcolor[rgb]{0.13,0.29,0.53}{\textbf{#1}}}
\newcommand{\NormalTok}[1]{#1}
\newcommand{\OperatorTok}[1]{\textcolor[rgb]{0.81,0.36,0.00}{\textbf{#1}}}
\newcommand{\OtherTok}[1]{\textcolor[rgb]{0.56,0.35,0.01}{#1}}
\newcommand{\PreprocessorTok}[1]{\textcolor[rgb]{0.56,0.35,0.01}{\textit{#1}}}
\newcommand{\RegionMarkerTok}[1]{#1}
\newcommand{\SpecialCharTok}[1]{\textcolor[rgb]{0.81,0.36,0.00}{\textbf{#1}}}
\newcommand{\SpecialStringTok}[1]{\textcolor[rgb]{0.31,0.60,0.02}{#1}}
\newcommand{\StringTok}[1]{\textcolor[rgb]{0.31,0.60,0.02}{#1}}
\newcommand{\VariableTok}[1]{\textcolor[rgb]{0.00,0.00,0.00}{#1}}
\newcommand{\VerbatimStringTok}[1]{\textcolor[rgb]{0.31,0.60,0.02}{#1}}
\newcommand{\WarningTok}[1]{\textcolor[rgb]{0.56,0.35,0.01}{\textbf{\textit{#1}}}}
\usepackage{longtable,booktabs,array}
\usepackage{calc} % for calculating minipage widths
% Correct order of tables after \paragraph or \subparagraph
\usepackage{etoolbox}
\makeatletter
\patchcmd\longtable{\par}{\if@noskipsec\mbox{}\fi\par}{}{}
\makeatother
% Allow footnotes in longtable head/foot
\IfFileExists{footnotehyper.sty}{\usepackage{footnotehyper}}{\usepackage{footnote}}
\makesavenoteenv{longtable}
\usepackage{graphicx}
\makeatletter
\def\maxwidth{\ifdim\Gin@nat@width>\linewidth\linewidth\else\Gin@nat@width\fi}
\def\maxheight{\ifdim\Gin@nat@height>\textheight\textheight\else\Gin@nat@height\fi}
\makeatother
% Scale images if necessary, so that they will not overflow the page
% margins by default, and it is still possible to overwrite the defaults
% using explicit options in \includegraphics[width, height, ...]{}
\setkeys{Gin}{width=\maxwidth,height=\maxheight,keepaspectratio}
% Set default figure placement to htbp
\makeatletter
\def\fps@figure{htbp}
\makeatother
\setlength{\emergencystretch}{3em} % prevent overfull lines
\providecommand{\tightlist}{%
  \setlength{\itemsep}{0pt}\setlength{\parskip}{0pt}}
\setcounter{secnumdepth}{-\maxdimen} % remove section numbering
\ifLuaTeX
  \usepackage{selnolig}  % disable illegal ligatures
\fi
\IfFileExists{bookmark.sty}{\usepackage{bookmark}}{\usepackage{hyperref}}
\IfFileExists{xurl.sty}{\usepackage{xurl}}{} % add URL line breaks if available
\urlstyle{same}
\hypersetup{
  pdftitle={Week-13-document},
  pdfauthor={Ho Zhi Yi},
  hidelinks,
  pdfcreator={LaTeX via pandoc}}

\title{Week-13-document}
\author{Ho Zhi Yi}
\date{2023-11-14}

\begin{document}
\maketitle

\hypertarget{week-9}{%
\section{\texorpdfstring{\textbf{Week 9}}{Week 9}}\label{week-9}}

\begin{enumerate}
\def\labelenumi{\arabic{enumi}.}
\item
  What is the topic that you have finalized? (Answer in 1 or 2
  sentences)

  \begin{itemize}
  \item
    \textbf{Topic:} Greenhouse Gas Emissions Trends in the World
  \item
    Analyzing Greenhouse Gas Emissions Trends based on:
  \item
    industry
  \end{itemize}

\begin{Shaded}
\begin{Highlighting}[]
\FunctionTok{distinct}\NormalTok{(data,Industry)}
\end{Highlighting}
\end{Shaded}

\begin{verbatim}
##                                                               Industry
## 1                                    Agriculture, Forestry and Fishing
## 2                                                         Construction
## 3                  Electricity, Gas, Steam and Air Conditioning Supply
## 4                                                        Manufacturing
## 5                                                               Mining
## 6                                            Other Services Industries
## 7                                                     Total Households
## 8                                        Total Industry and Households
## 9                                           Transportation and Storage
## 10 Water supply; sewerage, waste management and remediation activities
\end{verbatim}

  \begin{itemize}
  \tightlist
  \item
    region
  \end{itemize}

\begin{Shaded}
\begin{Highlighting}[]
\FunctionTok{distinct}\NormalTok{(data,Country)}
\end{Highlighting}
\end{Shaded}

\begin{verbatim}
##                              Country
## 1                 Advanced Economies
## 2                             Africa
## 3                           Americas
## 4                               Asia
## 5          Australia and New Zealand
## 6                       Central Asia
## 7                       Eastern Asia
## 8                     Eastern Europe
## 9  Emerging and Developing Economies
## 10                            Europe
## 11                               G20
## 12                                G7
## 13   Latin America and the Caribbean
## 14                   Northern Africa
## 15                  Northern America
## 16                   Northern Europe
## 17                           Oceania
## 18                South-eastern Asia
## 19                     Southern Asia
## 20                   Southern Europe
## 21                Sub-Saharan Africa
## 22                      Western Asia
## 23                    Western Europe
## 24                             World
\end{verbatim}

  \begin{itemize}
  \tightlist
  \item
    gas types
  \end{itemize}

\begin{Shaded}
\begin{Highlighting}[]
\FunctionTok{distinct}\NormalTok{(data,Gas\_Type)}
\end{Highlighting}
\end{Shaded}

\begin{verbatim}
##            Gas_Type
## 1    Carbon dioxide
## 2 Fluorinated gases
## 3    Greenhouse gas
## 4           Methane
## 5     Nitrous oxide
\end{verbatim}
\item
  What are the data sources that you have curated so far? (Answer 1 or 2
  sentences)

  \begin{itemize}
  \tightlist
  \item
    Annual greenhouse gas emissions by activity and by region (2010 to
    2021)
  \end{itemize}
\end{enumerate}

\hypertarget{week-10}{%
\section{\texorpdfstring{\textbf{Week 10}}{Week 10}}\label{week-10}}

\begin{enumerate}
\def\labelenumi{\arabic{enumi}.}
\item
  What is the question that you are going to answer? (Answer: One
  sentence that ends with a question mark that could act like the title
  of your data story)

  \begin{itemize}
  \tightlist
  \item
    Who contributed the most to greenhouse gas emissions?
  \end{itemize}
\item
  Why is this an important question? (Answer: 3 sentences, each of which
  has some evidence, e.g., \emph{``According to the United Nations...''}
  to justify why the question you have chosen is important),

  \begin{itemize}
  \item
    \textbf{Environmental Impact:} According to the United Nations,
    understanding the factors driving greenhouse gas emissions is
    crucial to address climate change and its environmental impact.
    Reducing emissions is essential to mitigate the consequences of
    global warming, such as more frequent extreme weather events and
    rising sea levels.
  \item
    \textbf{Economic Implications:} According to Earth.org, reducing
    carbon emissions would decrease the number of deaths related to air
    pollution and help to ease pressure on healthcare systems.
  \item
    \textbf{Policy and Mitigation:} Accoriding to the International
    Monetary Fund (IMF), to accelerate cuts to emissions, policymakers
    need detailed statistics to assist them in devising effective
    mitigation measures that can deliver the fastest and least
    disruptive pathway toward net zero emissions.
  \end{itemize}
\item
  Which rows and columns of the dataset will be used to answer this
  question? (Answer: Actual names of the variables in the dataset that
  you plan to use).

  \begin{itemize}
  \item
    \textbf{Columns:}

    \begin{itemize}
    \item
      \textbf{Year:} To track changes over time.

      \begin{longtable}[]{@{}
        >{\centering\arraybackslash}p{(\columnwidth - 22\tabcolsep) * \real{0.0833}}
        >{\centering\arraybackslash}p{(\columnwidth - 22\tabcolsep) * \real{0.0833}}
        >{\centering\arraybackslash}p{(\columnwidth - 22\tabcolsep) * \real{0.0833}}
        >{\centering\arraybackslash}p{(\columnwidth - 22\tabcolsep) * \real{0.0833}}
        >{\centering\arraybackslash}p{(\columnwidth - 22\tabcolsep) * \real{0.0833}}
        >{\centering\arraybackslash}p{(\columnwidth - 22\tabcolsep) * \real{0.0833}}
        >{\centering\arraybackslash}p{(\columnwidth - 22\tabcolsep) * \real{0.0833}}
        >{\centering\arraybackslash}p{(\columnwidth - 22\tabcolsep) * \real{0.0833}}
        >{\centering\arraybackslash}p{(\columnwidth - 22\tabcolsep) * \real{0.0833}}
        >{\centering\arraybackslash}p{(\columnwidth - 22\tabcolsep) * \real{0.0833}}
        >{\centering\arraybackslash}p{(\columnwidth - 22\tabcolsep) * \real{0.0833}}
        >{\centering\arraybackslash}p{(\columnwidth - 22\tabcolsep) * \real{0.0833}}@{}}
      \toprule\noalign{}
      \endhead
      \bottomrule\noalign{}
      \endlastfoot
      F2010 & F2011 & F2012 & F2013 & F2014 & F2015 & F2016 & F2017 &
      F2018 & F2019 & F2020 & F2021 \\
      \end{longtable}
    \item
      \textbf{Industry:} To examine emissions trends by sector.
    \item
      \textbf{Country:} To analyse regional variations in emissions.
    \item
      \textbf{Gas\_Type:} To understand the contributions of different
      greenhouse gases.
    \end{itemize}
  \item
    \textbf{Rows:}

    \begin{itemize}
    \item
      for Industry:

      \begin{itemize}
      \item
        Manufacturing
      \item
        Electricity, Gas, Steam and Air Conditioning Supply
      \item
        Transportation and Storage
      \item
        Agriculture, Forestry and Fishing
      \item
        Construction
      \end{itemize}
    \item
      for Country:

      \begin{itemize}
      \item
        Africa
      \item
        Americas
      \item
        Asia
      \item
        Europe
      \item
        Oceania
      \end{itemize}
    \item
      for Gas\_Type:

      \begin{itemize}
      \item
        Carbon dioxide
      \item
        Fluorinated gases
      \item
        Methane
      \item
        Nitrous oxide
      \end{itemize}
    \end{itemize}
  \end{itemize}
\item
  \emph{Include the challenges and errors that you faced and how you
  overcame them.}

  \begin{itemize}
  \item
    \textbf{Challenge 1}

    \begin{itemize}
    \item
      the regions available in the \textbf{Country column} contains some
      regions that overlap, hence I had to select suitable regions to
      ensure that there are no overlapping countries in the regions used
      for analysis
    \item
      the industries available in the \textbf{Industry column} contains
      some industries that overlap, hence I had to select suitable
      industries to ensure that there are no overlapping industries used
      for analysis
    \item
      \textbf{\emph{solution}}: select relevant variables in the Country
      and Industry columns through filtering the data followed by
      creating a new data frame by selecting the relevant columns as
      shown below
    \end{itemize}
  \end{itemize}

\begin{Shaded}
\begin{Highlighting}[]
\NormalTok{new\_data }\OtherTok{\textless{}{-}}\NormalTok{ data }\SpecialCharTok{\%\textgreater{}\%}
  \FunctionTok{filter}\NormalTok{(Country }\SpecialCharTok{\%in\%} \FunctionTok{c}\NormalTok{(}\StringTok{"Africa"}\NormalTok{, }\StringTok{"Americas"}\NormalTok{, }\StringTok{"Asia"}\NormalTok{, }\StringTok{"Europe"}\NormalTok{, }\StringTok{"Oceania"}\NormalTok{)) }\SpecialCharTok{\%\textgreater{}\%}
  \FunctionTok{filter}\NormalTok{(Industry }\SpecialCharTok{\%in\%} \FunctionTok{c}\NormalTok{(}\StringTok{"Manufacturing"}\NormalTok{, }\StringTok{"Electricity, Gas, }
\StringTok{                         Steam and Air Conditioning Supply"}\NormalTok{, }\StringTok{"Transportation and Storage"}\NormalTok{, }
                         \StringTok{"Agriculture, Forestry and Fishing"}\NormalTok{, }\StringTok{"Construction"}\NormalTok{)) }\SpecialCharTok{\%\textgreater{}\%}
  \FunctionTok{select}\NormalTok{(Country, Industry, Gas\_Type, F2010, F2011, F2012, F2013, F2014, F2015, }
\NormalTok{         F2016, F2017, F2018, F2019, F2020, F2021) }
\end{Highlighting}
\end{Shaded}

  \begin{itemize}
  \item
    \textbf{Challenge 2}

    \begin{itemize}
    \item
      column names were not ideal and straight forward to use
    \item
      the individual years were not named in numerical form
    \item
      the Country column was a bit confusing because the values under
      this column are regions rather than specific countries
    \item
      \textbf{\emph{solution}}: rename the columns accordingly as shown
      below
    \end{itemize}
  \end{itemize}

\begin{Shaded}
\begin{Highlighting}[]
\NormalTok{new\_names }\OtherTok{\textless{}{-}} \FunctionTok{c}\NormalTok{(}\StringTok{"Region"}\NormalTok{, }\StringTok{"Industry"}\NormalTok{, }\StringTok{"Gas\_Type"}\NormalTok{, }\DecValTok{2010}\NormalTok{, }\DecValTok{2011}\NormalTok{, }\DecValTok{2012}\NormalTok{, }\DecValTok{2013}\NormalTok{, }\DecValTok{2014}\NormalTok{, }\DecValTok{2015}\NormalTok{, }
               \DecValTok{2016}\NormalTok{, }\DecValTok{2017}\NormalTok{, }\DecValTok{2018}\NormalTok{, }\DecValTok{2019}\NormalTok{, }\DecValTok{2020}\NormalTok{, }\DecValTok{2021}\NormalTok{)}

\NormalTok{new\_data }\SpecialCharTok{\%\textgreater{}\%} \FunctionTok{set\_names}\NormalTok{(new\_names)}
\end{Highlighting}
\end{Shaded}
\end{enumerate}

\hypertarget{week-11}{%
\section{Week 11}\label{week-11}}

\begin{enumerate}
\def\labelenumi{\arabic{enumi}.}
\item
  List the visualisations that you are going to use in your project
  (Answer: What are the variables that you are going to plot? How will
  it answer your larger question?)

  \begin{itemize}
  \item
    \textbf{Line graphs} of

    \begin{itemize}
    \item
      industry against time
    \item
      each industry against time using shiny
    \item
      region against time
    \item
      each region against time using shiny
    \item
      gas type against time
    \item
      each gas type against time using shiny
    \end{itemize}
  \item
    \textbf{Bar graphs} of

    \begin{itemize}
    \item
      each gas type against industry using shiny
    \item
      each gas type against region using shiny
    \end{itemize}
  \item
    These graphs will assist in pinpointing the region, industry, and
    specific type of gas that exerts the most significant influence on
    the greenhouse gas effect, thereby providing valuable information
    for crafting policies targeted at reducing the primary source of
    emissions.
  \end{itemize}
\item
  How do you plan to make it interactive? (Answer: features of
  ggplot2/shiny/markdown do you plan to use to make the story
  interactive)

  \begin{itemize}
  \tightlist
  \item
    Create shiny apps to allow users to select various variables to view
    the plots for the chosen variable
  \item
    Use ggplot2 with plotly to make the plots interactive
  \end{itemize}
\item
  What concepts incorporated in your project were taught in the course
  and which ones were self-learnt? (Answer: Create a table with topics
  in one column and Weeks in the other to indicate which concept taught
  in which week is being used. Leave the entry of the Week column empty
  for self-learnt concepts)

  \begin{longtable}[]{@{}
    >{\centering\arraybackslash}p{(\columnwidth - 4\tabcolsep) * \real{0.0465}}
    >{\raggedright\arraybackslash}p{(\columnwidth - 4\tabcolsep) * \real{0.8953}}
    >{\raggedright\arraybackslash}p{(\columnwidth - 4\tabcolsep) * \real{0.0523}}@{}}
  \toprule\noalign{}
  \begin{minipage}[b]{\linewidth}\centering
  No.
  \end{minipage} & \begin{minipage}[b]{\linewidth}\raggedright
  Concepts
  \end{minipage} & \begin{minipage}[b]{\linewidth}\raggedright
  Week
  \end{minipage} \\
  \midrule\noalign{}
  \endhead
  \bottomrule\noalign{}
  \endlastfoot
  1 & Reading csv file

  Accessing elements of a vector & Week 3 \\
  2 & \begin{minipage}[t]{\linewidth}\raggedright
  Manipulating data

  \begin{itemize}
  \item
    tidy data ( filter( ), summarise( ) )
  \item
    choosing row/column
  \item
    piping operators (\%\textgreater\%)
  \item
    calculating mean and sum
  \end{itemize}
  \end{minipage} & Week 4 \\
  3 & \begin{minipage}[t]{\linewidth}\raggedright
  Functions

  \begin{itemize}
  \tightlist
  \item
    writing functions
  \end{itemize}
  \end{minipage} & Week 5 \\
  4 & \begin{minipage}[t]{\linewidth}\raggedright
  Iterations

  \begin{itemize}
  \tightlist
  \item
    for loops
  \end{itemize}
  \end{minipage} & Week 6 \\
  5 & \begin{minipage}[t]{\linewidth}\raggedright
  Visualizing data using Shiny ggplot2

  \begin{itemize}
  \tightlist
  \item
    creating line and bar graphs
  \end{itemize}
  \end{minipage} & Week 7 \\
  6 & \begin{minipage}[t]{\linewidth}\raggedright
  Visualizing data using Shiny

  \begin{itemize}
  \tightlist
  \item
    creating Shiny app for line and bar graphs
  \end{itemize}
  \end{minipage} & Week 8 \\
  7 & \begin{minipage}[t]{\linewidth}\raggedright
  Exploratory Data Analyses

  \begin{itemize}
  \tightlist
  \item
    pivot\_longer( ) function
  \end{itemize}
  \end{minipage} & Week 9 \\
  8 & \begin{minipage}[t]{\linewidth}\raggedright
  \textbf{Self-learnt}

  \begin{itemize}
  \item
    using kable( ) to display the sample data in a table form
  \item
    sapply( ) function
  \item
    creating styles.scss file to edit the theme of the website as it
    allows me to define variables, making it easier to edit the colours
    of the website
  \item
    using panel-tabset to create tabs to show the plot and code in
    different tabs
  \item
    using plotly to make the plots interactive (including plots in shiny
    app)
  \end{itemize}
  \end{minipage} & - \\
  \end{longtable}
\item
  \emph{Include the challenges and errors that you faced and how you
  overcame them.}

  \begin{itemize}
  \item
    \textbf{Challenge 1}

    \begin{itemize}
    \item
      values of variables were not in the correct data type, hence
      returning errors when generating the plots
    \item
      \textbf{\emph{solution}}: convert the values to specific data
      types using

      \begin{itemize}
      \tightlist
      \item
        as.character(), as.numeric(), as.integer()
      \end{itemize}
    \end{itemize}
  \item
    \textbf{Challenge 2}

    \begin{itemize}
    \item
      the data is too wide, making it difficult to manipulate for
      calculations
    \item
      \textbf{\emph{solution}}: use pivot\_longer() to reshape the data
      to a longer format
    \end{itemize}
  \end{itemize}
\end{enumerate}

\hypertarget{week-12}{%
\section{Week 12}\label{week-12}}

\begin{enumerate}
\def\labelenumi{\arabic{enumi}.}
\tightlist
\item
  \emph{Include the challenges and errors that you faced and how you
  overcame them (if any)}

  \begin{itemize}
  \item
    \textbf{Challenge 1}

    \begin{itemize}
    \item
      the height of the bar graphs plotted may be misleading as the y
      axis is auto adjusted to fit the values for each graph
    \item
      \textbf{\emph{solution}}: incorporate plotly into all the plots to
      show the actual emissions value and make it interactive at the
      same time
    \end{itemize}
  \item
    \textbf{Challenge 2}

    \begin{itemize}
    \item
      in the 4th shiny app of plotting the graphs for each gas type
      against industry, the industry names were too long, so they are
      overlapping one another. also, the legend is too long hence when
      incorporating plotly into the graphs, the graphs are shrunk to the
      point that they can't be seen
    \item
      \textbf{\emph{solution}}: use the first letter of each industry as
      the labels to make it shorter using substring()
    \end{itemize}
  \end{itemize}
\end{enumerate}

\hypertarget{week-13-final-submission}{%
\section{Week 13 (Final Submission)}\label{week-13-final-submission}}

\hypertarget{diary-entry}{%
\subsection{Diary Entry}\label{diary-entry}}

\begin{enumerate}
\def\labelenumi{\arabic{enumi}.}
\item
  What is the theme of your data story?

  \begin{itemize}
  \tightlist
  \item
    The theme of the data story revolves around understanding and
    addressing greenhouse gas emissions trends from 2010 to 2021. The
    analysis focuses on industries, regions, and gas types to identify
    key contributors to greenhouse gas emissions. The question that we
    are trying to answer is: Who is the highest contributor?
  \end{itemize}
\item
  Why is it important to address this question?

  \begin{itemize}
  \tightlist
  \item
    Greenhouse gas emissions is a primary driver of climate change.
    Identifying the most significant source of emissions allows
    policymakers, scientists, and the public to make informed decisions
    and implement targeted strategies for mitigation. The urgency of
    addressing this question stems from the immediate and long-term
    impacts of climate change on ecosystems, economies, and human
    well-being.
  \end{itemize}
\item
  Why do you think the data sources that you have curated can help you
  answer the question?

  \begin{itemize}
  \item
    \textbf{Credibility of the International Monetary Fund (IMF):} The
    emissions data was obtained from the IMF website. The IMF is a
    reputable international organization that collects and disseminates
    economic and financial data. It is known for its rigorous data
    collection processes and commitment to providing accurate and
    reliable information. This enhances the credibility of the
    greenhouse gas emissions data obtained.
  \item
    \textbf{Comprehensive Coverage:} The dataset covers a range of
    industries, regions, and gas types, providing a comprehensive view
    of greenhouse gas emissions. This allows holistic analysis of the
    data.
  \item
    \textbf{Standardised Unit of Measurement:} The use of a standardised
    unit of measurement (million metric tons of CO2 equivalent) ensures
    consistency and facilitates meaningful comparisons across different
    categories. This is crucial for accurate analysis and interpretation
    of the emissions data.
  \end{itemize}
\item
  What are the insights from the data and how are they depicted in
  plots?

  \begin{itemize}
  \item
    \textbf{Line Graphs}: Highlight emissions trends (2010-2021) for
    industries, regions, and gas types

    \begin{itemize}
    \tightlist
    \item
      Identified that the Electricity, Gas, Steam, and Air Conditioning
      Supply industry, Asia, and Carbon Dioxide are the highest
      contributors to greenhouse gas emissions based on industry, region
      and gas type respectively.
    \end{itemize}
  \item
    \textbf{Bar Graphs}: Visualize the total emissions of each gas type
    across industries and regions

    \begin{itemize}
    \tightlist
    \item
      Proved that the Electricity, Gas, Steam, and Air Conditioning
      Supply industry in Asia is the highest contributor to carbon
      dioxide gas emissions.
    \end{itemize}
  \end{itemize}
\item
  How did you implement this entire project? Were there any new concepts
  that you learnt to implement some aspects of it?

  \begin{itemize}
  \item
    \textbf{Data Cleaning:} Cleaned the dataset to ensure that the
    dataset is suitable for analysis
  \item
    \textbf{Data Visualisation:} Utilised line graphs and bar graphs to
    visualise the emissions data
  \item
    \textbf{New concepts:} Integrated plotly into the ggplot2 plots and
    shiny app to make the plots interactive
  \end{itemize}
\end{enumerate}

\hypertarget{write-up}{%
\subsection{Write-up}\label{write-up}}

\hypertarget{background}{%
\subsubsection{Background}\label{background}}

Greenhouse gas emissions, primarily from human activities, drive climate
change by trapping heat in the atmosphere, impacting human health and
ecosystems (Basics of Climate Change \textbar{} US EPA, 2023). The
global focus on greenhouse gas emissions has increased, with 54\% of the
world's population considering climate change a very serious problem
(Budiman, 2020). Recognising the environmental consequences, there is a
pressing need to understand and address the intricacies of these
emissions.

This growing acknowledgment underscores the necessity to study and
analyse greenhouse gas emissions data. As societies grapple with their
carbon footprint, a deeper understanding of the factors driving
emissions is crucial for informed decision-making and effective
strategies to mitigate the impacts of climate change.

In this writeup, we will delve into an analysis of greenhouse gas
emission trends covering the period from 2010 to 2021. The focus will be
to examine the contributions of different industries, regions, and gas
types to greenhouse gas emissions. The dataset used for this analysis
was sourced from the International Monetary Fund, ensuring a credible
and reliable foundation for the investigation. The primary objective is
to uncover the highest contributor to greenhouse gas emissions by
meticulously discerning patterns and trends within the provided data.
This analysis seeks to provide valuable insights that can guide
strategies and decisions aimed at tackling the challenges presented by
the most significant greenhouse gas emissions.

\hypertarget{approach}{%
\subsubsection{Approach}\label{approach}}

To comprehensively analyse greenhouse gas emission trends from 2010 to
2021, the approach involves a systematic examination of data focusing on
industries, regions, and gas types. The following steps outlines the
methodology:

\begin{enumerate}
\def\labelenumi{\arabic{enumi}.}
\tightlist
\item
  Data Collection:
\end{enumerate}

The dataset, encompassing annual greenhouse gas emissions data from 2010
to 2021, was procured from the International Monetary Fund, which is a
renowned source known for its precision and reliability. The information
is categorised based on industry, region, and gas type.

Link to IMF dataset:
\url{https://climatedata.imf.org/datasets/c8579761f19740dfbe4418b205654ddf}

\begin{enumerate}
\def\labelenumi{\arabic{enumi}.}
\setcounter{enumi}{1}
\tightlist
\item
  Cleaning and Formatting:
\end{enumerate}

To ensure precision in the analysis, relevant variables related to
industry, region, gas type, and yearly emissions (2010 to 2021) were
filtered from the dataset, ensuring that only data essential to the
investigation was retained for further analysis. Also, a critical aspect
of data clarity lies in proper naming and formatting. Therefore, the
column names were renamed to be clear, reflecting the nature of the data
they encapsulate. This enhances the data's readability and at the same
time make it easier to manipulate. Lastly, due to the fact that there
were overlapping industries and regions, selection of suitable
industries and regions was needed.

\begin{enumerate}
\def\labelenumi{\arabic{enumi}.}
\setcounter{enumi}{2}
\tightlist
\item
  Exploratory Data Analysis:
\end{enumerate}

Various statistical and graphical techniques were employed to extract
insights into the overarching patterns and distributions of greenhouse
gas emissions data from the dataset. Key visualisations including line
graphs and bar graphs were generated.

\begin{enumerate}
\def\labelenumi{\arabic{enumi}.}
\setcounter{enumi}{3}
\tightlist
\item
  Interactive Visualisations:
\end{enumerate}

Through the utilisation of plotly, ggplot2 and shiny app, the
visualisations offered interactivity for exploring greenhouse gas
emissions data. Users can zoom, hover, and focus on specific data
points, ensuring a dynamic and engaging experience. This approach
encourages active user involvement in the exploration process, enhancing
the understanding of greenhouse gas emissions trends.

\begin{enumerate}
\def\labelenumi{\arabic{enumi}.}
\setcounter{enumi}{4}
\tightlist
\item
  Identification of Highest Contributor of Greenhouse Gas Emissions:
\end{enumerate}

Through an in-depth analysis of data trends from the visualisations, the
goal was to pinpoint the industry, region, and gas type that emerged as
the highest contributor of greenhouse gas emissions spanning from 2010
to 2021.

Subsequently, bar graphs were used to prove if the identified industry
and region were indeed the highest contributors to the specific gas type
associated with the highest emissions.

\begin{enumerate}
\def\labelenumi{\arabic{enumi}.}
\setcounter{enumi}{5}
\tightlist
\item
  Insights and Recommendations:
\end{enumerate}

Extracted insights from the line and bar graphs provided a comprehensive
understanding of the dynamics of greenhouse gas emissions. Through
recognising potential shortcomings in the analysis, there is room for
improvements to enhance accuracy of the insights. These insights forms
the groundwork for potential recommendations and strategies, addressing
the challenges posed by the identified highest contributor.

\hypertarget{final-outcome}{%
\subsubsection{Final Outcome}\label{final-outcome}}

The visualisations led to a conclusive finding that the primary
contributor to greenhouse gas emissions is the Electricity, Gas, Steam,
and Air Conditioning Supply industry in Asia, particularly associated
with carbon dioxide emissions. The line graphs distinctly illustrated
that this identified industry, Asia region, and carbon dioxide gas type
exhibited the highest contributions to greenhouse gas emissions spanning
from 2010 to 2021. Furthermore, the supporting evidence from the bar
graphs substantiates the initial identification, providing additional
confirmation that the Electricity, Gas, Steam, and Air Conditioning
Supply industry in Asia is indeed the highest contributor to carbon
dioxide gas emissions.

Therefore, with this conclusion, suggested policies and strategies can
be directed towards mitigating the impact of carbon dioxide emissions
from the Electricity, Gas, Steam, and Air Conditioning Supply industry
in Asia, contributing to a more effective approach in addressing the
challenges posed by greenhouse gas emissions.

\hypertarget{references}{%
\subsubsection{References}\label{references}}

Basics of climate Change \textbar{} \emph{US EPA}. (2023, November 1).
US EPA.
\url{https://www.epa.gov/climatechange-science/basics-climate-change\#}

Budiman, A. (2020, July 27). \emph{Global Concern about Climate Change,
Broad Support for Limiting Emissions} \textbar{} \emph{Pew Research
Center}. Pew Research Center's Global Attitudes Project.
\url{https://www.pewresearch.org/global/2015/11/05/global-concern-about-climate-change-broad-support-for-limiting-emissions/\#}

\emph{Climate Change Indicators Dashboard}. (2021, March 19).
\url{https://climatedata.imf.org/datasets/c8579761f19740dfbe4418b205654ddf/explore}

\end{document}
